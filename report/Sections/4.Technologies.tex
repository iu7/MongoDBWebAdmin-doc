\section{Технологическая часть}

\subsection{Выбор технологий}
Ввиду того что было решено создавать веб-приложение для администрирования с базой данных, то в качестве языка разметки был выбран язык HTML и CSS для управления стилями веб страницы. Для реализации динамических частей веб страницы использовался JavaScript. Со стороны сервера запросы обрабатываются фреймвороком Flask. В качестве СУБД была выбрана MongoDB. Доступ к базе данных осуществляется на стороне сервевра с использованием библиотеки PyMongo, что позволяет получать доступ к MongoDB приложением, написанным на Python.\par
Для создания расчетно-пояснительной записки использовалась система компьютерной верстки LaTeX.\par
Для улучшения процесса разработки програмного продукта использовалась система контроля верий Git. Для этого был создан репозиторий на сайте GitHub.com. Для улучшения взаимействия между участниками команды была использована гибкая методолгия разработки програмного обеспечения (Agile).

\subsection{MongoDB}
MongoDB — документо-ориентированная СУБД с открытым исходным кодом, не требующая описания схемы таблиц. За счёт минимизации семантики для работы с транзакциями появляется возможность решения целого ряда проблем, связанных с недостатком производительности, причём горизонтальное масштабирование становится проще. Используемая модель документов хранения данных (JSON/BSON) проще кодируется, проще управляется (в том числе за счёт применения «бессхемного стиля», а внутренняя группировка релевантных данных обеспечивает дополнительный выигрыш в быстродействии.\par
MongoDB заполняет разрыв между простейшими NoSQL-СУБД, хранящими данные в виде «ключ — значение» (простыми и легко масштабируемыми, но обладающими минимальными функциональными возможностями) и большими реляционными СУБД (со структурными схемами и мощными запросами).\par

Основные возможности MongoDB:
\begin{itemize}
    \item Документо-ориентированное хранение (JSON-подобная схема данных)
    \item Достаточно гибкий язык для формирования запросов
    \item Динамические запросы
    \item Поддержка индексов
    \item Профилирование запросов
    \item Эффективное хранение двоичных данных больших объёмов, например, фото и видео
    \item Журналирование операций, модифицирующих данные в базе данных
    \item Поддержка отказоустойчивости и масштабируемости: асинхронная репликация, набор \item реплик и распределения базы данных на узлы
    \item Может работать в соответствии с парадигмой MapReduce
    \item Полнотекстовый поиск, в том числе на русском языке, с поддержкой морфологии
\end{itemize}

\subsection{Flask}
Для обработки запросов на стороне сервера используется фреймворк Flask.
Flask - легкий фреймворк для веб приложений, написанный на Python. Flask использует технологию WSGI (Web Server Gateway Interface) и использует движок Jinja2 для обработки шаблонов. Является бесплатным и распостраняется по лицензии BSD.\par
Flask сочетает в себе гибкость языка программирования Python и простоту использования шаблонов для веб разрботки. Flask является микрофреймворком, поэтмоу предоставляет небольшую функциональность, однако легко расширяется.\par

Особенности Flask:
\begin{itemize}
    \item Сервер для разработки и отладки
    \item Интегрированная поддержка юнит тестирования
    \item Обработка запросов RESTful
    \item Использование Jinja2 для развертывания шаблонов
    \item Поддержка безопасных cookie (со стороны сессии клиента)
    \item Полная поддержка стандарта WSGI 1.0
    \item Поддерджка Unicode
    \item Совместимость с Google App Engine
    \item Поддержка дополнительных расширений
\end{itemize}

\subsection{LaTeX}
\subsection{Git and GitHub}
\subsection{Agile}

\newpage
