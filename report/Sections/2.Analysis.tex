\section{Аналитический раздел}


\subsection{NoSQL базы данных}

На ряду с реляционными базами данных (SQL) существуют так же и нереляционные (NoSQL).
\par
Традиционные СУБД ориентируются на требования ACID к транзакционной системе: атомарность, согласованность, изолированность, надёжность, тогда как в NoSQL вместо ACID может рассматриваться набор свойств BASE:

\begin{itemize}
    \item базовая доступность - каждый запрос гарантированно завершается (успешно или безуспешно).
    \item гибкое состояние - состояние системы может изменяться со временем, даже без ввода новых данных, для достижения согласования данных.
    \item согласованность в конечном счёте - данные могут быть некоторое время рассогласованы, но приходят к согласованию через некоторое время.
\end{itemize}

\par
NoSQL СУБД имеют такие отличительные черты, как возможность разработки базы данных без задания схемы и таблиц, линейная маштабируемость, быстрый отклик системы, превосходная поддержка многопроцессорности.
\par
В данной лабораторной работе было решено создавать инструмент для администрации именно для нереляционных баз данных, ввиду их преимуществ и популярности использования в современных системах. В качестве NoSQL базы данных была выбрана докумкнто-ориентированная СУБД MongoDB.


\subsection{Документо-ориентированная СУБД}

Документо-ориентированная СУБД — СУБД, специально предназначенная для хранения иерархических структур данных (документов) и обычно реализуемая с помощью подхода NoSQL. В основе документо - ориентированных СУБД лежат документные хранилища, имеющие структуру дерева (иногда леса). Структура дерева начинается с корневого узла и может содержать несколько внутренних и листовых узлов. Листовые узлы содержат данные, которые при добавлении документа заносятся в индексы, что позволяет даже при достаточно сложной структуре находить путь искомых данных. API для поиска позволяет находить по запросу документы и части документов. В отличие от хранилищ типа ключ-значение, выборка по запросу к документному хранилищу может содержать части большого количества документов без полной загрузки этих документов в оперативную память.\par
Документы могут быть сгруппированы в коллекции. Их можно считать отдалённым аналогом таблиц реляционных СУБД, но коллекции могут содержать другие коллекции. Хотя документы коллекции могут быть произвольными, для более эффективного индексирования лучше объединять в коллекцию документы с похожей структурой.


\subsection{Сравнение существующих систем администрирования}

Для обзора уже существующих систем администрирования были выбраны три, одни из  наиболее популярных, из них.

Результаты сравнения представлены в таблице №1.

\textit{Таблица №1.}
\begin{center}
\begin{tabular}{| L{5cm} | C{3cm} | C{3cm} | C{3cm} |}
    \hline
    \diagbox{~~Функции~~~}{Системы}
     ~ & Robomongo & Umongo & mViewer \\ \hline
    Командная оболочка (shell) & \Checkmark & \XSolidBrush & \Checkmark \\ \hline
    Простые операции БД (create, drop, command, eval, \ldots) & \Checkmark & \Checkmark & Create, Drop \\ \hline
    Простые операции над коллекциями (update, duplicate, remove, \ldots) & \Checkmark & \Checkmark & Create, Remove, Update, Редактирование JSON напрямую \\ \hline
    Операции с индексами (create, remove, \ldots) & \Checkmark & \Checkmark & \XSolidBrush \\ \hline
    Операции для работы с шардингом (enable, add shard, shard collection, \ldots) & Только получение статистики и коллекций & \Checkmark & Только просмотр коллекций \\ \hline
    Статистика сервера & \Checkmark & \Checkmark & \Checkmark \\ \hline
    Платформа & Windows, Linux, Mac OS X & Windows, Linux, Mac OS X & Windows, Linux, Mac OS X \\ \hline
\end{tabular}
\end{center}


\newpage
